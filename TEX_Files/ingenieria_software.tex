\documentclass[a4paper,11pt]{article}
\usepackage[utf8]{inputenc}
\usepackage{enumitem}
\usepackage{lipsum}
\usepackage{fancyhdr}
\usepackage{titlesec}
\usepackage{setspace}  % Para espaciado
\usepackage[margin=1in]{geometry}  % Para reducir márgenes

% Ajustes de formato
\setstretch{1}  % Espaciado entre líneas
\setlength{\parskip}{0.5em}  % Espaciado entre párrafos

\title{Ingeniería de Software: Resumen}
\author{UADE}
\date{}

\pagestyle{fancy}
\fancyhf{}
\fancyfoot[C]{\thepage}

\begin{document}

% Carátula
\maketitle
\thispagestyle{empty}
\newpage

% Clase 1
\section*{Clase 1: Introducción a la Ingeniería de Software}

\textbf{Introducción a la Ingeniería de Software (IS)}: La ingeniería de software (IS) es la disciplina que se encarga del desarrollo, operación y mantenimiento del software mediante un enfoque sistemático, disciplinado y cuantificable. Este campo surge como respuesta a la creciente complejidad de los sistemas de software y su importancia en la sociedad moderna.

\textbf{Importancia de la IS}: La IS es crucial porque permite crear software de alta calidad, fiable y seguro, que cumple con los requisitos de los usuarios y se adapta a sus necesidades cambiantes. Sin un enfoque estructurado, el software puede presentar fallas, ser costoso de mantener o resultar difícil de evolucionar.

\textbf{Evolución de la IS}: Desde sus inicios, la IS ha pasado de ser un proceso rudimentario a una disciplina madura que utiliza metodologías ágiles, integración continua y prácticas de DevOps, garantizando ciclos de desarrollo más rápidos y eficientes.

\textbf{Ejemplo práctico}: Imagina que una empresa debe desarrollar una aplicación de banca en línea. Sin aplicar principios de ingeniería de software, la aplicación podría ser vulnerable a ataques, tener tiempos de inactividad prolongados y resultar insatisfactoria para los clientes.

\textbf{Estado actual y tendencias}: Hoy en día, la ingeniería de software ha incorporado tecnologías emergentes como la inteligencia artificial (IA), el desarrollo ágil, la automatización y el uso de la nube, lo que facilita la implementación de software más rápidamente y con mayor calidad.

\textbf{Proceso del software}: El ciclo de vida del desarrollo de software incluye varias fases clave: análisis de requerimientos, diseño, implementación, pruebas, despliegue y mantenimiento. Cada una de estas fases es crítica para asegurar que el producto final funcione según lo esperado y sea escalable en el futuro.

\textbf{Definición IEEE}: Según la IEEE, la IS es "la aplicación de un enfoque sistemático, disciplinado y cuantificable al desarrollo, operación y mantenimiento del software". Esta definición subraya la necesidad de un enfoque formal en todas las etapas del ciclo de vida del software.

\textbf{Bibliografía}: Pressman, R. (Cap. 1) \textit{Ingeniería del Software, Un enfoque práctico}.

\newpage

% Clase 2
\section*{Clase 2: Dominios de Aplicación y Procesos de Ingeniería}

\textbf{Dominios de Aplicación del Software}:

El software puede clasificarse en diferentes dominios de aplicación, cada uno con características y requisitos específicos. A continuación, se detallan algunos de los más relevantes:

\begin{enumerate}[label=\arabic*.]

    \item \textbf{Software de sistemas}: Comprende sistemas operativos, controladores de dispositivos y otros programas que gestionan directamente el hardware de la computadora. Ejemplo: Windows, Linux.
    \item \textbf{Software de aplicación}: Incluye programas que ayudan al usuario a realizar tareas específicas, como procesadores de texto, hojas de cálculo o navegadores web. Ejemplo: Microsoft Word, Google Chrome.
    \item \textbf{Software de ingeniería y ciencias}: Se utiliza para realizar cálculos avanzados y simulaciones en campos como la ingeniería y las ciencias. Ejemplo: MATLAB, AutoCAD.
    \item \textbf{Software incrustado}: Software embebido en dispositivos electrónicos para controlar su funcionamiento. Se encuentra en electrodomésticos, automóviles, dispositivos médicos, entre otros. Ejemplo: Software en un sistema ABS de un automóvil.
    \item \textbf{Software de línea de productos}: Plataformas de software reutilizables que permiten la creación de diferentes productos en una misma familia, optimizando el desarrollo. Ejemplo: Variantes de teléfonos móviles de una misma marca.
    \item \textbf{Aplicaciones web (Webapps)}: Software accesible a través de un navegador, como plataformas de comercio electrónico, aplicaciones de redes sociales o herramientas colaborativas. Ejemplo: Google Drive, Amazon.
    \item \textbf{Software de inteligencia artificial (IA)}: Sistemas que pueden aprender, adaptarse y tomar decisiones basadas en datos. Se utiliza en áreas como el reconocimiento de voz, visión por computadora, y sistemas de recomendación. Ejemplo: Alexa, sistemas de recomendación en Netflix.

\end{enumerate}

\textbf{Ejemplo práctico}: En el sector automotriz, el software incrustado permite que los vehículos modernos funcionen de manera más eficiente y segura. Desde el sistema de frenado ABS hasta la conectividad de entretenimiento, todo está controlado por software.

\textbf{Procesos de ingeniería}: Para desarrollar software, se sigue un conjunto de procesos fundamentales:
\begin{itemize}
    \item \textbf{Comunicación}: Entender las necesidades del cliente o usuario.
    \item \textbf{Planeación}: Establecer un plan para el desarrollo, tiempos y recursos.
    \item \textbf{Modelado}: Definir la estructura y el diseño del software.
    \item \textbf{Construcción}: Implementar el software.
    \item \textbf{Despliegue}: Entregar el software al usuario final y asegurar su correcta instalación y funcionamiento.
\end{itemize}

\newpage

\textbf{Actividades sombrilla}: Estas actividades se realizan a lo largo de todo el proceso de desarrollo:
\begin{itemize}
    \item \textbf{Control de proyecto}: Monitorear el progreso del proyecto para asegurar que se cumplan los objetivos.
    \item \textbf{Gestión de riesgos}: Identificar posibles problemas y desarrollar estrategias para mitigarlos.
    \item \textbf{Control de calidad}: Asegurar que el producto final cumple con los estándares establecidos.
    \item \textbf{Revisiones técnicas}: Evaluar el software en diferentes fases para identificar posibles fallos.
    \item \textbf{Administración de la configuración}: Gestionar los cambios en el software de manera controlada.
\end{itemize}

\textbf{Bibliografía}: Pressman, R. (Cap. 1).

\newpage

% Clase 3
\section*{Clase 3: Scripting y Automatización}

\textbf{Instalación de paquetes de software}: Existen diversos métodos para instalar software en sistemas complejos:
\begin{itemize}
    \item \textbf{Instalación directa}: Se instala el software de una sola vez.
    \item \textbf{Instalación en paralelo}: Se instala el nuevo software junto con la versión anterior para comparar resultados.
    \item \textbf{Instalación piloto}: Se implementa en una pequeña parte del sistema antes de hacer el despliegue completo.
    \item \textbf{Instalación en fases}: Se instala gradualmente, probando el software en diferentes etapas del proceso.
\end{itemize}

\textbf{Scripting}: 
\begin{itemize}
    \item Los \textbf{scripts} son programas pequeños diseñados para automatizar tareas repetitivas, mientras que los \textbf{programas} son más complejos y estructurados. Los scripts pueden ejecutarse sin necesidad de compilación y son usados comúnmente para automatizar procesos rutinarios.
    \item \textbf{Tipos de scripts}: Línea de comandos, scripts para aplicaciones web y scripts de automatización que ayudan a ejecutar tareas administrativas en servidores o entornos de trabajo.
\end{itemize}

\textbf{Ejemplo práctico}: Los administradores de sistemas usan scripts de línea de comandos para realizar copias de seguridad automáticas o reiniciar servicios cuando un servidor falla.

\textbf{Balanceo de carga}: Es una técnica que distribuye las solicitudes de usuarios entre varios servidores para garantizar que ninguno esté sobrecargado. Algunos métodos incluyen:
\begin{itemize}
    \item \textbf{Round Robin}: Distribuye las solicitudes de manera equitativa entre todos los servidores.
    \item \textbf{Least connections}: Envía las nuevas solicitudes al servidor con menos conexiones activas.
    \item \textbf{IP hashing}: Asigna solicitudes basadas en la IP del cliente.
\end{itemize}

\textbf{Reliability (Confiabilidad)}: La confiabilidad de un sistema se mide a través de dos conceptos importantes:
\begin{itemize}
    \item \textbf{MTBF (Mean Time Between Failures)}: El tiempo promedio entre fallos.
    \item \textbf{MTTR (Mean Time to Repair)}: El tiempo promedio que toma reparar un fallo.
\end{itemize}

\newpage

% Clase 4
\section*{Clase 4: Pruebas y Operaciones}

\textbf{Migración de Datos}:
\begin{itemize}
    \item \textbf{Planificación}: La migración debe planearse cuidadosamente, mapeando los datos desde el sistema origen al destino, asegurando la consistencia de las relaciones y estructuras.
    \item \textbf{Ejecución}: Una vez completada la migración, se realizan pruebas exhaustivas para verificar la integridad y calidad de los datos.
\end{itemize}

\textbf{Ejemplo práctico}: En una migración de datos de una base de datos local a la nube, la planificación minuciosa es esencial para evitar la pérdida de información crítica.

\textbf{Gestión de Cambios y Problemas}:
\begin{itemize}
    \item \textbf{Gestión de cambios}: Documentar y autorizar cualquier modificación en el software antes de implementarla.
    \item \textbf{Gestión de problemas}: Registrar, analizar y resolver incidentes que afecten la operación del sistema.
\end{itemize}

\textbf{Control de Operaciones}: Monitorear los sistemas en tiempo real para asegurar que operen correctamente, identificar problemas rápidamente y responder de manera oportuna.

\textbf{Bibliografía}: Pressman, R. (Cap. 11).

\newpage

% Clase 5
\section*{Clase 5: Proceso de Tercerización}

\textbf{Tercerización (BPO)}: La tercerización o Business Process Outsourcing (BPO) consiste en delegar procesos o servicios a un proveedor externo especializado. Esto permite a las empresas enfocarse en sus actividades principales mientras el proveedor externo se encarga de tareas no esenciales como soporte técnico, procesamiento de nóminas o atención al cliente.

\textbf{Service Level Agreement (SLA)}: Un acuerdo de nivel de servicio (SLA) es un contrato que establece los estándares de calidad, disponibilidad y responsabilidad entre el proveedor y el cliente. Define parámetros clave como el tiempo de respuesta ante problemas y la calidad del servicio ofrecido.

\textbf{Ejemplo práctico}: Una empresa que terceriza su soporte técnico firma un SLA con un proveedor externo, el cual especifica que cualquier problema debe ser atendido en menos de 4 horas para asegurar una operación fluida.

\textbf{Software Factory}: Este concepto se refiere a una metodología industrializada para la creación de software, donde se emplean técnicas de automatización y mejora de procesos para aumentar la productividad y calidad del producto final. Las fábricas de software permiten gestionar varios proyectos en paralelo y escalar el desarrollo.

\textbf{Bibliografía}: Varios recursos sobre BPO y SLAs.

\end{document}
