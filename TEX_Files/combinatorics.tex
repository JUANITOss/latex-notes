\documentclass{article}
\usepackage{amsmath}
\usepackage{amsfonts}
\usepackage{amssymb}

\title{Combinatorics: Understanding Permutations, Variations, and Combinations}
\author{by Juan Suazo Verger}
\date{}

\begin{document}

\maketitle

\newpage
\section{Permutations (No Repetition)}

\subsection{Definition}
Permutations count the number of ways to arrange all elements of a set where \textbf{order matters} and \textbf{no element repeats}.

\subsection{Formula}
\[
P(n) = n!
\]
Where \(n\)! (n factorial) is the product of all positive integers up to \(n\).

\subsection{Example}
Imagine you have 3 different books and you want to arrange them on a shelf. The number of possible arrangements is:
\[
P(3) = 3! = 3 \times 2 \times 1 = 6
\]
Arrangements: (Book1, Book2, Book3), (Book1, Book3, Book2), etc.

\subsection{When to Apply}
- Use permutations when you need to arrange all items in a specific order.
- Example scenarios: Arranging people in a line, shuffling cards.

\newpage
\section{Permutations (With Repetition)}

\subsection{Definition}
Permutations with repetition allow \textbf{elements to be repeated} in the arrangement. The order still matters.

\subsection{Formula}
\[
P(n, r) = n^r
\]
Where \(n\) is the number of distinct items, and \(r\) is the number of positions.

\subsection{Example}
Imagine you have 3 different letters (A, B, C) and you want to create a 2-letter code, where letters can repeat. The number of possible 2-letter codes is:
\[
P(3, 2) = 3^2 = 9
\]
Possible codes: AA, AB, AC, BA, BB, BC, etc.

\subsection{When to Apply}
- When order matters and elements can repeat.
- Example scenarios: Creating codes or passwords where characters can repeat.

\newpage
\section{Variations (No Repetition)}

\subsection{Definition}
Variations (or arrangements) deal with selecting and ordering a subset of elements, where \textbf{order matters}, but \textbf{elements are not repeated}.

\subsection{Formula}
\[
V(n, r) = \frac{n!}{(n - r)!}
\]
Where \(n\) is the total number of elements, and \(r\) is the number of elements to arrange.

\subsection{Example}
Imagine you have 5 people and you want to select 3 of them to form a committee with specific roles. The number of ways to assign the roles is:
\[
V(5, 3) = \frac{5!}{(5 - 3)!} = \frac{5!}{2!} = \frac{120}{2} = 60
\]
Possible combinations: (Person1, Person2, Person3), (Person2, Person3, Person1), etc.

\subsection{When to Apply}
- Use when you are selecting a subset and arranging it in a specific order.
- Example scenarios: Choosing teams for different roles (e.g., leader, assistant, etc.).

\newpage
\section{Variations (With Repetition)}

\subsection{Definition}
This allows selecting and arranging a subset where \textbf{order matters}, and \textbf{elements can repeat}.

\subsection{Formula}
\[
V(n, r) = n^r
\]

\subsection{Example}
Imagine you have 4 different colors and want to paint 3 rooms, where each room can be painted the same or a different color. The number of ways to paint the rooms is:
\[
V(4, 3) = 4^3 = 64
\]
Possible combinations: (Red, Red, Blue), (Green, Yellow, Green), etc.

\subsection{When to Apply}
- Use when selecting a subset with order and allowing repetition.
- Example scenarios: Choosing ice cream flavors for scoops where the same flavor can be used multiple times.

\newpage
\section{Combinations (No Repetition)}

\subsection{Definition}
Combinations are used when \textbf{order doesn’t matter} and \textbf{elements cannot repeat}.

\subsection{Formula}
\[
C(n, r) = \binom{n}{r} = \frac{n!}{r!(n - r)!}
\]
Where \(n\) is the total number of elements, and \(r\) is the number of elements to select.

\subsection{Example}
Imagine you have 5 different fruits and want to pick 3 for a fruit salad, where the order in which you pick them doesn’t matter. The number of ways to choose the fruits is:
\[
C(5, 3) = \frac{5!}{3!(5 - 3)!} = \frac{120}{6 \times 2} = 10
\]
Possible combinations: (Apple, Banana, Cherry), (Apple, Banana, Grape), etc.

\subsection{When to Apply}
- Use combinations when selecting a subset without regard to order.
- Example scenarios: Choosing lottery numbers, selecting a team from a group where roles don’t matter.

\newpage
\section{Combinations (With Repetition)}

\subsection{Definition}
Combinations with repetition allow \textbf{elements to be repeated}, but \textbf{order does not matter}.

\subsection{Formula}
\[
C(n + r - 1, r) = \frac{(n + r - 1)!}{r!(n - 1)!}
\]

\subsection{Example}
Imagine you have 3 types of cookies (chocolate, vanilla, strawberry) and want to select 4 cookies, allowing repeats. The number of ways to choose the cookies is:
\[
C(3 + 4 - 1, 4) = C(6, 4) = \frac{6!}{4!2!} = \frac{720}{24 \times 2} = 15
\]
Possible combinations: (2 chocolate, 1 vanilla, 1 strawberry), etc.

\subsection{When to Apply}
- Use when you need to select a subset allowing repetition but without regard to order.
- Example scenarios: Dividing items into groups where duplicates are allowed (e.g., scooping flavors for ice cream).

\newpage
\section{Symmetry in Combinations}

The \textbf{binomial coefficient} \(\binom{n}{k}\) has an important property called \textbf{symmetry}. This is expressed as:

\[
\binom{n}{k} = \binom{n}{n-k}
\]

This means that the number of ways to choose \(k\) elements from a set of \(n\) elements is the same as the number of ways to choose \(n-k\) elements.

\subsection{Example}
If \(n = 5\) and \(k = 2\), then:
\[
\binom{5}{2} = \binom{5}{3} = 10
\]

\newpage
\section{Alternative Notations for Combinations}

Combinations are often represented in different notations. For example:

\[
\binom{n}{k} = C_k^n = C(n, k)
\]

All of these notations represent the same concept: choosing \(k\) elements from \(n\).

- \(\binom{n}{k}\) is the standard binomial coefficient notation.

- \(C_k^n\) is an alternative notation.

- \(C(n, k)\) is another alternative used in some contexts.

\newpage
\section{How to Recognize and Apply Each Formula}

\begin{itemize}
  \item \textbf{Order Matters?} 
    \begin{itemize}
      \item Yes: Use \textbf{permutations} or \textbf{variations}.
      \item No: Use \textbf{combinations}.
    \end{itemize}
  \item \textbf{Repetition Allowed?}
    \begin{itemize}
      \item Yes: Use \textbf{permutations} or \textbf{combinations} \textbf{with repetition}.
      \item No: Use \textbf{permutations} or \textbf{combinations} \textbf{without repetition}.
    \end{itemize}
\end{itemize}

\end{document}
