\documentclass{article}
\usepackage{listings}
\usepackage{xcolor}

\definecolor{codegray}{gray}{0.9}
\lstset{
    backgroundcolor=\color{codegray},
    basicstyle=\ttfamily,
    columns=fullflexible,
    keepspaces=true,
    frame=single,
    breaklines=true,
    keywordstyle=\bfseries,
    morekeywords={SELECT, FROM, WHERE, AS, BETWEEN, CURDATE, DATE_ADD, DATE_SUB, TIMESTAMPDIFF, DAYOFWEEK, NOW, CURTIME, AVG, GROUP BY},
}

\begin{document}

\section*{SQL Date Manipulation Cheat Sheet}

\begin{lstlisting}
1. Extracting Parts of a Date:
   SELECT YEAR(date_column) AS year,
          MONTH(date_column) AS month,
          DAY(date_column) AS day
   FROM table_name;

   SELECT HOUR(datetime_column) AS hour,
          MINUTE(datetime_column) AS minute,
          SECOND(datetime_column) AS second
   FROM table_name;

   SELECT WEEK(date_column) AS week_number FROM table_name;
   SELECT DAYOFWEEK(date_column) AS day_of_week FROM table_name;
   SELECT DAYOFYEAR(date_column) AS day_of_year FROM table_name;
   SELECT QUARTER(date_column) AS quarter FROM table_name;

2. Adding/Subtracting Dates:
   SELECT DATE_ADD(date_column, INTERVAL 5 DAY) AS future_date FROM table_name;
   SELECT DATE_SUB(date_column, INTERVAL 5 DAY) AS past_date FROM table_name;
   SELECT DATE_ADD(date_column, INTERVAL 2 MONTH) AS future_date FROM table_name;
   SELECT DATE_ADD(datetime_column, INTERVAL 3 HOUR) AS future_time FROM table_name;

3. Comparing Dates:
   SELECT * FROM table_name WHERE date_column < CURDATE();  -- past
   SELECT * FROM table_name WHERE date_column > CURDATE();  -- future
   SELECT * FROM table_name WHERE date_column = CURDATE();  -- today
   SELECT * FROM table_name 
   WHERE date_column BETWEEN '2024-01-01' AND '2024-12-31'; -- date range

4. Formatting Dates:
   SELECT DATE_FORMAT(date_column, '%Y-%m-%d') AS formatted_date FROM table_name;
   -- Example: '%d/%m/%Y %H:%i:%s' would return '02/10/2024 14:30:45'

5. Calculating the Difference Between Dates:
   SELECT DATEDIFF('2024-10-10', '2024-01-01') AS days_difference;
   SELECT TIMESTAMPDIFF(MONTH, '2023-01-01', '2024-10-10') AS months_difference;
   SELECT TIMESTAMPDIFF(HOUR, '2024-01-01 08:00:00', '2024-01-01 18:00:00') AS hours_difference;

6. Getting the Current Date and Time:
   SELECT CURDATE() AS current_date;
   SELECT CURTIME() AS current_time;
   SELECT NOW() AS current_datetime;

7. Working with Weekdays and Weekends:
   SELECT * FROM table_name WHERE DAYOFWEEK(date_column) IN (1, 7); -- weekend
   SELECT * FROM table_name WHERE DAYOFWEEK(date_column) BETWEEN 2 AND 6; -- weekday

8. Truncating Dates:
   SELECT DATE_FORMAT(date_column, '%Y-%m-01') AS first_of_month FROM table_name;
   SELECT DATE(datetime_column) AS truncated_date FROM table_name;

9. Date Arithmetic with INTERVAL:
   SELECT DATE_ADD(date_column, INTERVAL 2 WEEK) AS future_date FROM table_name;
   SELECT LAST_DAY(date_column) AS last_day_of_month FROM table_name;
   SELECT MAKEDATE(YEAR(CURDATE()), 1) AS first_day_of_year;
\end{lstlisting}

\end{document}
